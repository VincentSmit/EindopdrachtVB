\chapter{Appendix}
\section{Lexer specificatie}
Voor de lexer zijn de verschillende tokens van belang. Deze tokens staan hieronder allen gedefinieerd. \\ 

Tekens
\begin{description}
    \item \itab{keyword}    \tab{tesken}
    \item \itab{COLON}      \tab{:}
    \item \itab{SEMICOLON}  \tab{;}
    \item \itab{LPAREN}     \tab{(} 
    \item \itab{RPAREN}     \tab{)} 
    \item \itab{LBLOCK}     \tab{[}
    \item \itab{RBLOCK}     \tab{]}
    \item \itab{LCURLY}     \tab{\{}
    \item \itab{RCURLY}     \tab{\}}
    \item \itab{COMMA}      \tab{,}
    \item \itab{DOUBLE\_QUOTE} \tab{"}
    \item \itab{SINGLE\_QUOTE} \tab{\textbackslash}
    \item \itab{BODY}       \tab{body}
\end{description}

\clearpage

Operators
\begin{description}
    \item \itab{keyword}    \tab{teken}
    \item \itab{PLUS}       \tab{+}
    \item \itab{MINUS}      \tab{-}
    \item \itab{DIVIDES}    \tab{/}
    \item \itab{MULTIPL}    \tab{*}
    \item \itab{POWER}      \tab{\^}
    \item \itab{LT}         \tab{\textless}
    \item \itab{GT}         \tab{\textgreater}
    \item \itab{GTE}        \tab{\textgreater =}
    \item \itab{LTE}        \tab{\textless =}
    \item \itab{EQ}         \tab{=}
    \item \itab{NEQ}        \tab{!}
    \item \itab{ASSIGN}     \tab{==}
    \item \itab{OR}         \tab{\textbar \textbar}
    \item \itab{AND}        \tab{\&\&}
\end{description}

\clearpage

Keywords van KIDEB
\begin{description}
    \item \itab{keyword}    \tab{keyword in de taal}
    \item \itab{PROGRAM}    \tab{program}
    \item \itab{SWAP}       \tab{swap}
    \item \itab{IF}         \tab{if}
    \item \itab{THEN}       \tab{then}
    \item \itab{ELSE}       \tab{else}
    \item \itab{DO}         \tab{do}
    \item \itab{WHILE}      \tab{while}
    \item \itab{FROM}       \tab{from}
    \item \itab{IMPORT}     \tab{import}
    \item \itab{BREAK}      \tab{break}
    \item \itab{CONTINUE}   \tab{continue}
    \item \itab{RETURN}     \tab{return}
    \item \itab{FOR}        \tab{for}
    \item \itab{IN}         \tab{in}
    \item \itab{RETURNS}    \tab{returns}
    \item \itab{FUNC}       \tab{func}
    \item \itab{ARRAY}      \tab{array}
    \item \itab{ARGS}       \tab{args}
    \item \itab{VAR}        \tab{var}
    \item \itab{OF}         \tab{of}
\end{description}

\clearpage

Standaard types
\begin{description}
    \item \itab{keyword}    \tab{keyword in de taal}
    \item \itab{INTEGER}    \tab{int}
    \item \itab{CHARACTER}  \tab{char}
    \item \itab{BOOLEAN}    \tab{bool}
    \item \itab{CALL}       \tab{call}
\end{description}

\clearpage
\section{Parser specificatie}

\clearpage
\section{Testen}

\clearpage
\chapter{Testplan en resultaten} % (fold)
\label{cha:testplan_en_resultaten}
Om de juistheid van de compiler te garanderen, is testen essentieel. Hiertoe zijn dan ook twee typen tests geschreven. De eerste is een in \emph{python} geschreven programma, welke de syntax en context test. Het tweede type bestaat uit grotere voorbeeldprogramma's, welke syntax, context en semantiek testen.

De testprogramma's geschreven in \emph{python} zijn te vinden in het bijgeleverde zip-bestand in de map \emph{tests}. De testprogramma's uit onze eigen taal in de map \emph{examples}.

\section{Syntax- en contexttests} % (fold)
\label{sec:syntax_en_context}
De python programma's testen zeer snel en effici\"ent de volledige syntax en context. Hieronder vallen zowel correcte als foute code. Als de code bewust fout gaat, wordt dit afgevangen op de correcte foutmelding. 

Als al deze tests slagen is er geen noemenswaardige output. Mocht er een test wel falen, dan wordt dit getoond op de standaard output. 

% section syntax_en_context (end)

\section{Semantiektests} % (fold)
\label{sec:semantiektests}

De voorbeeldprogramma's in de map \emph{tests} testen de semantiek van de taal. Deze programma's leveren TAM-code en ... op als output. Het grootste testbestand is bijgevoegd in de appendix in sectie \ref{sec:testverslag}.

\begin{description}
    \item[array.kib] Het declareren van, toewijzen van waarden aan, en uitlezen van een array.
    \item[power.kib] Het declareren en gebruik van een functie.
    \item[fibonacci\_recursive.kib] Het gebruik van recursie.
    \item[heap.kib] Het alloceren en vrijgeven van geheugen wordt hier getest. 
    \item[pointers.kib] Test de declaratie en het gebruik van pointers.
    \item[raw.kib] Het direct uitvoeren van TAM-code in de KIDEB broncode.
    \item[all.kib] Test alle functionaliteit van de taal. Programma en ouput is te vinden in sectie \ref{sec:testverslag}.
\end{description}
% section semantiektests (end)

% chapter testplan_en_resultaten (end)
\clearpage
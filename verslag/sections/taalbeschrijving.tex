\chapter{Taalbeschrijving}
    \label{chap:taalbeschrijving}
KIDEB is een kleine, imperatieve programmeertaal met beperkte mogelijkheden, ontwikkeld als project voor het vak vertalerbouw als onderdeel van de Technische Informatica bachelor van de Universiteit Twente. Ondanks de beperkte mogelijkheden bevat de taal toch enkele leuke onderdelen.

Uiteraard is de basis van een programmeertaal ook aanwezig in KIDEB. Variabelen kunnen worden gedeclareerd met primitieve typen: integer, boolean en character. Door toegevoegde type inferentie hoeft het primitieve type zels niet expliciet te worden vermeld. 

Op deze variabelen zijn een aantal bewerkingen mogelijk. Zo zijn de standaard boolean operaties beschikbaar, om vershil of gelijkheid te bepalen. Daarnaast bevat de taal de basis rekenkundige operaties en zelfs bestaat de mogelijkheid tot machtsverheffen.

Belangrijke statements zijn ook ge\"implementeerd. If-else is onderdeel van de taal, evenals een while-statment. Voor het doorlopen van een complete array is het for-statement bijgevoegd.

De eerste belangrijke uitbreiding is de mogelijkheid gebruik te maken van subroutines. Naast het hoofdprogramma zijn namelijk ook functies te defini\"eren en uit te voeren. Functies hebben de mogelijkheid tot het opleveren van een waarde, maar dat is niet verplicht.

De tweede belangrijke uitbreiding was al even genoemd. Dit is namelijk het gebruik van arrays. Door het gebruik van arrays wordt de taal een stuk complexer, maar ook een stuk sterker.

De volledige grammatica van de taal is te vinden in het hoofdstuk \ref{chap:syntax_context_en_semantiek}.

\clearpage
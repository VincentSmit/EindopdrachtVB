\chapter{Vertaalregels} % (fold)
\label{cha:vertaalregels}
Om de vertaalregels van de broncode naar TAM-code duidelijk te maken, worden code templates gebruikt. KIDEB kent de volgende acties.

\begin{tabular}{| l | l | p{6 cm} |}
    \hline
    Klasse & Code functie & Effect van gegenereerde code \\
    \hline
    Program & \emph{run P} & Draai programma P en daarna stoppen. Beginnen en eindigen met een lege stack. \\
    \hline
    Commando & \emph{do C} & Is de volledige lijst met instructies, bestaande uit statements, expressies en declaraties. \\
    \hline
    Statement & \emph{execute S} & Voer het statement S uit met mogelijk aanpassen van variabelen, maar zonder effect op de stack. \\
    \hline
    Expressie & \emph{evaluate E} & Evalueer de expressie E en push het reultaat naar de stack. Geen verder effect op de stack. \\
    \hline
    Identifier & \emph{fetch I} & Push de waarde van identifier I naar de stack. \\
    \hline
    Identifier & \emph{assign I} & Pop een waarde van de stack en sla deze op in variabele I. \\
    \hline
    Declaratie & \emph{declare D} & Verwerk declaratie D, breidt de stack uit om ruimte te maken voor variabelen die hierin gedeclareerd worden. \\
    \hline
\end{tabular} 

Een programma in KIDEB is een serie commands. Elk los command kan een statement, een expressie of een declaratie zijn. Een programma kan er dus als volgt uitzien.
\begin{description}
    \item[run [[C]]] = \hfill \\
        do C \\
        HALT
\end{description}

Dit leidt tot de volgende commando's.
\begin{description}
    \item[do [[S]]] = \hfill \\
        execute S
    \item[do [[E]]] = \hfill \\
        evaluate E
    \item[do [[D]]] = \hfill \\
        declare D 
\end{description}

De statements gaan als volgt.
\begin{description}
    \item[execute [[I = E;]]] \hfill \\
        evaluate E \\
        assign I
    \item[execute [[if E then C\textsubscript{1} else C\textsubscript{2}]]] = \hfill \\
        \begin{tabular}{c l}
            &evaluate E \\
            &JUMPIF(0) \emph{g} \\
            &execute C\textsubscript{1} \\
            &JUMP \emph{h} \\
    \emph{g:}&gexecute C\textsubscript{2} \\
    \emph{h:}& \\
        \end{tabular}
    \item[execute [[while E then C]]] = \hfill \\
        \begin{tabular}{c l}
            &JUMP \emph{h} \\
    \emph{g:}&execute C \\
    \emph{h:}&evaluate E \\
            &JUMPIF(1) \emph{g} \\
        \end{tabular} 
    \item[execute [[return E]]]  = \hfill \\
        evaluate E \\
        RETURNS(1)
    \item[execute [[import SW]]] = \hfill \\ 
        Is wel een statement, maar levert geen code op. Importeert nl een bestand met naam SW, en voegt de AST van die code toe aan de AST van het hoofdbestand.
    \item[execute [[print E]]] = \hfill \\
        evaluate E \\
        CALL put \\
        \itab{LOADL 10}\stab{newline} \\
        CALL putint
\end{description}

De expressie worden als volgt verwerkt.
\begin{description}
    \item[evaluate [[E\textsubscript{1} O E\textsubscript{2}]]] = \hfill \\
        evaluate E\textsubscript{1} \\
        evaluate E\textsubscript{2} \\
        \itab{CALL p}\stab{p is het adres van de primitieve routine die hoort bij O}
    \item[evaluate [[I]]] = \hfill \\
        fetchV
    \item[evaluate [[]]] 
\end{description}

Laden en opslaan van waarden in variabelen
\begin{description}
    \item[fetch [[I]]] 
    \item[assign [[I]]]
\end{description}

Declaraties worden hiermee afgehandeld.
\begin{description}
    \item[declare [[T I]]] = \hfill \\ 

\end{description}
% chapter vertaalregels (end)
\clearpage
\chapter{Vertaalregels} % (fold)
\label{cha:vertaalregels}
Om de vertaalregels van de broncode naar TAM-code duidelijk te maken, worden code templates gebruikt. KIDEB kent de volgende acties.

\begin{tabular}{| l | l | p{6 cm} |}
    \hline
    Klasse & Code functie & Effect van gegenereerde code \\
    \hline
    Program & \emph{run P} & Draai programma P en daarna stoppen. Beginnen en eindigen met een lege stack. \\
    \hline
    Statement & \emph{execute E} & Voer het statement S uit met mogelijk aanpassen van variabelen, maar zonder effect op de stack. \\
    \hline
    Expressie & \emph{evaluate E} & Evalueer de expressie E en push het reultaat naar de stack. Geen verder effect op de stack. \\
    \hline
    Identifier & \emph{fetch I} & Push de waarde van identifier I naar de stack. \\
    \hline
    Identifier & \emph{assign I} & Pop een waarde van de stack en sla deze op in variabele I. \\
    \hline
    Declaration & \emph{declare D} & Verwerk declaratie D, breidt de stack uit om ruimte te maken voor variabelen die hierin gedeclareerd worden. \\
    \hline
\end{tabular} 

Een programma in KIDEB is een serie commands. Elk los command kan een statement, een expressie of een declaratie zijn. Een programma kan er dus als volgt uitzien.
\begin{description}
    \item[run [[S]]] = \hfill \\
        execute S \\
        HALT
    \item[run [[E]]] = \hfill \\
        evaluate E \\
        HALT
    \item[run [[D]]] = \hfill \\
        declare D \\
        HALT
\end{description}
% chapter vertaalregels (end)
\clearpage